\section{Adversary Model}
%%In this section, we explore the attacks that can be carried out by the adversary. We explain our infrastructure which we used to carry out attack on our victim. Finally, we present the taxonomy of attacks with discussion on their viability, stealthiness and  implications for the domain owners.

%MK: In this section, we present the taxonomy of attacks by our adversary using DNS dynamic updates on vulnerable slave server. 
In this section, we present the taxonomy of attacks by an adversary on a vulnerable authoritative name server.
We refer to the attacks exploiting non-secure DNS dynamic updates as to \textit{zone poisoning}.
%
We first explain our setup to perform lab experiments, followed by 
presenting the attack vectors tested in the lab environment. 
%the methodology which we used to %compromise domain validation to obtain a valid digital certificate for a vulnerable domain.
%confirm different attack vectors.
%Finally, w
We discuss the viability and stealthiness of these attacks and %MK: also explain the implications for the domain owners. 
the implications for owners of vulnerable domains.

%using zone poisoning vulnerability. We discuss the viability and stealthiness of these attacks and also explain the implications for the domain owners.

\subsection{Infrastructure Setup}

In order to explain the adversary model and test it in an isolated environment, we set up our infrastructure %comprising of 
composed of two servers. 
%MK: A DNS server that is vulnerable to receive  dynamic zone updates (victim) without authentication from any authoritative server. The second server will act as an adversary to update the zone files of our victim.  We explain these configurations in detail below. 


\textbf{Adversary configuration:} Depending on the type of the attack, different configurations might be required. 
In our setup, we assume that our adversary has already scanned for vulnerable 
resources and knows the domain name and its authoritative name server.
%MK: servers and knows the public IP address of the victim. 
The adversary will also require the standard dynamic DNS update utility \texttt{nsupdate}\footnote{%Dynamic DNS update utility:
\url{https://linux.die.net/man/8/nsupdate}} (or another dedicated software) to modify the zone of the victim. %Furthermore, f
For more advanced attacks the adversary might need to set up a mail, a web or a DNS server.
%Depending on the type of attack, different configurations might be required. In our setup, we assume our adversary has already scanned for vulnerable servers and knows the public IP address of our victim. 

%\hl{has access to the compromised machine} \michal{It's a bit confusing to me. Does it mean, that the adversary has like ssh access? Or just that there's connectivity between both hosts? Also the Victim is not yet compromised during the setup, right?} with a public IP address. The adversary will also require Dynamic DNS update utility \texttt{nsupdate} to modify the zone of our victim. Furthermore, for more sophisticated attacks other services like email server and webserver are required. %An email server with forwarding is also required for more sophisticated attacks explained later. EDIT

\textbf{Victim configuration:} Our victim server is running BIND 9.10.3 DNS and Apache HTTP software. 
%MK: For the adversary to modify DNS records, t
The victim name server (\url{ns1.example.com}) is configured to allow non-secure dynamic updates %of records as explained in the previous section. 
from anyone using the \texttt{allow-update} clause with the built-in argument ``any'' in the zone configuration of our domain (\url{example.com}). %, as explained in the previous section.
%
%We assume that o
Our victim is hosting a website, 
which is protected with an SSL certificate. 
The victim server might also have a mail server configured with the corresponding \texttt{MX} record defined in the zone~file.  


\subsection{Taxonomy of Attacks}
We define the attacks by the adversary in the following three categories and present the tradeoff between viability and stealthiness of each attack:

%We present the details of the following three types of attacks by the adversary and discuss the viability and stealthiness of these attacks. We also analyze the implications of these attacks on our victim

\begin{itemize}
\item \textbf{Denial of Service (DoS) attack:} In the DoS attack the adversary will be able to disrupt a service (e.g.: web, mail, domain, ssh, etc.) of the victim temporarily. 
It is much easier to execute with limited stealthiness, as the victim would notice the unresponsive domain and would be able to fix the configuration quickly. 


\item \textbf{Man-in-the-Middle (MITM) attack:} The MITM attack requires more sophistication from the adversary while being more stealthy and difficult to detect. In this type of the attack, %our 
the adversary will %comprise 
poison the victim's zone file and redirect traffic to its server and then back to the victim. % without getting detected. 
It can be used to capture the traffic between the victim's server and its clients. %customers.  

\item \textbf{Domain shadowing attack:} %Finally, t
The adversary %might also be able to 
can also spawn malicious subdomains which she can then use for exploit kits, malware, phishing and stealing information from other victims. 
Domain shadowing attacks~\cite{shadowing} are %more
very stealthy in nature. 
The subdomains inherit the trust of the parent domain which enables them to evade countermeasures based on domain reputation or blacklisting systems. 
The adversary can %spawn 
add multiple subdomains at no cost and can rotate between them to evade detection. 

The domain shadowing attack can be further sophisticated by by-passing domain control validation in order to setup a valid SSL certificate for a malicious subdomain.
%MK: In Section~\ref{sec:domainvalidation} we explain the details of this attack vector.

%Furthermore in Section~\ref{sec:domainvalidation} we explain how attackers can by-pass domain control validation in order to setup SSL certificates. 
\end{itemize}



\subsubsection{Denial of Service (DoS) Attack} 
The DoS attack will %only 
cause limited disruption in the service until the victim gets notified and fixes the problem. However, these types of attacks can result in financial losses due to domain downtime or loss of customers. The adversary can cause %a DoS attack 
 a DoS using one of the following methods: %methodologies. 

\textbf{Deletion of an \texttt{A}/\texttt{AAAA} record of a domain or a subdomain:} The adversary can remove an \texttt{A} or \texttt{AAAA} record which will delete the mapping between the domain and the corresponding IPv4 or IPv6 address of \url{example.com} from the zone file. 
%MK: It will result in downtime of the traffic. 
%%MK: Note that adversary can also delete a subdomain %with the same command.
%%MK: with \texttt{nusupdate} utility command ``\texttt{update delete example.com A}''. 
Note that adversary can also delete a subdomain, assuming that subdomains are managed by the same parent name server (\url{ns1.example.com}). 
%corresponding to the web, ftp, or mail service. 
If the victim is running different services like accounts, mail, ftp, or checkout on a separate subdomain, it will %partially 
disable a part of the website and might take more time to~detect. 
%with \texttt{nusupdate} utility command \texttt{update delete example.com A} using nsupdate. 

 \textbf{Deletion of a zone delegation:} 
%
 The parent name server (\url{ns1.example.com} in the tested case) may perform the delegation: the \url{example.com} zone to the \url{subdomain.example.com} zone maintained by the child name server (\url{ns1.subdomain.example.com}).
 %Delegated zones can be used for variety of purposes. It can be used for management of domains and subdomains or for the backup if master zone is down,  when configured as slave. 
 %If the adversary successfully deletes 
 The adversary %may 
 can delete the delegation in the zone file, i.e. the mapping between a subdomain and the corresponding child name server, or a glue record, i.e. an \texttt{A/AAAA} record providing the mapping between the name of the child server and its IP address.
 It will cause the service disruption of the subdomain(s) (in our case \url{subdomain.example.com}) handled by the delegated server. %, but would also disrupt a backup. 


\textbf{Deletion of an MX record:} Similarly, the adversary can also delete an \texttt{MX} (mail exchange) record (or its glue record) which specifies the name of the mail server responsible for accepting email messages on behalf of a domain name.
It will not only disrupt an email service but also hinder any abuse messages sent to the victim. 
% Deletion of NS records not possible 
%\textbf{Deletion of NS record:} Deletion of NS record will be most disruptive for the victim as all available domains including email service will be un-accessible to victim and its users till the zone file is updated with correct records. 
 % Need to check following statements...
% Check edit of SOA records .... 

\textbf{Deletion, addition or modification of a \texttt{TXT} record:} Text records are extensively used for implementation of the Sender Policy Framework (SPF) \cite{spf}, Domain Keys Identified Mail (DKIM) \cite{dkim}, Domain-based Message Authentication, Reporting and Conformance (DMARC) \cite{dmarc}, or DNS-based service discovery \cite{rfc6763}.
Due to the widespread use of the \texttt{TXT} record by a variety of services and therefore possible threats, we focus on the DoS attacks against SPF.

SPF is an email authentication method designed to detect spoofed sender addresses. %Qualifiers are critical elements of SPF.
For example, the %\texttt{v=spf1 mx -all} qualifier allows the domain's \texttt{MX} hosts to send mail for the domain \url{example.com}, 
\texttt{v=spf1 ip4:130.190.0.0/16} \texttt{ip6:2001:660:5301:24:5d79:9378/96 -all} qualifier %allows the %acceptable 
authorizes the specific IPv4 and IPv6 address ranges 
and prohibits all other hosts from sending emails for the domain \url{example.com}.
%
%MK: %to send mail for the domain \url{example.com} and prohibits all other hosts.
In the simplest form of the attack, the adversary can modify the \texttt{TXT} record to \texttt{v=spf1 -all} to prohibit sending emails from any hosts. %the valid hosts or any hosts at all. 
This attack is highly viable but the rapid detection of the poisoned \texttt{TXT} record reduces the stealthiness of the attack. %, if the victim identifies that the problem lies in misconfigured SPF record.
%The attacker might also delete the original record and create two SPF entries
The adversary can also delete the original \texttt{TXT} record or replace it with multiple \texttt{TXT} records corresponding to each of the SPF entries (in our example to two entries that authorize IPv4 and IPv6 address ranges). % ranges to send mail). % for \url{example.com}). 
If the record set includes more than one SPF entry the evaluation produces the  permanent error and the rejection of the mail is an option according to RFC 7208 \cite{spf}\footnote{In our experiments we do not examine how the email service providers implement the RFC specification.}.
%
%MK: Some of the recipient mail servers may decline all of them as only one \texttt{TXT} record with multiple SPF entries is allowed by the specification \cite{spf}. 
% And some will accept only one: https://stackoverflow.com/questions/19142369/create-both-ipv4-and-ipv6-spf-record
%
% https://serverfault.com/questions/29777/will-adding-a-second-spf-record-mess-up-my-dns
%
%
%This may increase the stealthiness as the admin would see correct entries.
Similarly, if the domain does not have a valid SPF record, %the result is a permanent error. %Some mail receivers
the evaluation also returns the permanent error and the mail may be rejected during the SMTP transaction \cite{spf}.
% https://community.spiceworks.com/topic/2140473-spf-record-syntax-error
%
%https://serverfault.com/questions/855639/spf-record-fails-how-to-verify-server-ip-address-and-managing-multiple-spf-rec
%Finally, note that an improper format of the SPF entry results in 
%If the domain does not have a valid SPF record, the result is a permanent error
%
%rejecting victim's email by the recipient server to reject victim's email.
Therefore, the attacker can modify %by deleting and adding 
the relevant \texttt{TXT} record with a syntax error, or a typo. 
For example, the required hyphen in the ``\texttt{-all}'' type could be replaced by the en-dash. %, which would increase the stealthiness of the attack in comparison to the previous methods. 
This method increases significantly the stealthiness of the attack in comparison to the previous methods. 
%https://community.spiceworks.com/topic/2140473-spf-record-syntax-error


\begin{comment}
\textbf{Deletion or modification of a \texttt{PTR} record:} 
A \texttt{PTR} record, or a reverse DNS of an IP, translates the IP address to its hostname.
The \texttt{PTR} record is used by different services, notably by e-mail anti-spam techniques to protect recipients from phishing attacks.
Most of receiving mail server checks if the IP address of the sending mail server has a \texttt{PTR} record and if it matches the IP obtained via the forward DNS resolution of the hostname defined in the \texttt{PTR} record.
Emails without the \texttt{PTR} record may get rejected or marked as spam, therefore, the adversary could modify or completely delete the \texttt{PTR} record to prevent recipients from receiving legitimate emails.

The tricky part is that the PTR record is not configured by the DNS service provider but by the ISP, so zone poisoning fails.
\end{comment}



\subsubsection{Man-in-the-Middle (MITM) Attacks}
The second type of the attack that adversary can use is the Man-in-the-Middle. 
These attacks are more sophisticated but also harder to detect. 

\textbf{Update of an \texttt{A/AAAA} record of a domain or a subdomain:}
An adversary can update\footnote{by deleting the original resource record and adding the new one.} an \texttt{A/AAAA} record of %\url{example.com} (and its subdomains) 
a domain or a subdomain in the zone file of the victim and change the existing IP address to the IP address of the machine under her control. %its compromised machine. 
%In this way the traffic directed towards the  domain of the victim will be redirected towards the adversary. % and will results in the domain hijacking.
In this way all the traffic will be redirected towards the adversary.
If the adversary acts as a simple proxy server between the victim and its clients and only passively observes the traffic then she can gain limited but useful information about, %the clients, 
%victims. For instance, 
%%such as 
for example, the number of clients, %customers, geo-location based on 
their IP addresses, or pages that were visited by the clients %customers 
(in case of web traffic). 
%However, for
%Depending on the service run by the victim on the vulnerable domain (web, ssh, ftp) the adversary needs to setup a corresponding server to intercept the communication and 
In a more sophisticated attack the adversary can setup a service, in particular a phishing website to steal %all the  personal 
critical information from the customers of the victim, such as logins and passwords. %, or credit card numbers. 
She should then forward the collected information back to the %customer 
victim to increase the stealthiness of the~attack. 

\textbf{Update of an \texttt{MX} record:} 
If the victim runs a mail server, the adversary can also change the \texttt{MX} RR (e.g. \texttt{MX  10 mail.example.com}) and replace \url{mail.example.com} with the name of the mail server maintained by her (e.g. \url{mail.malicious.com}).
The adversary may also need to set up an additional name server which will provide the DNS resolution of \url{mail.malicious.com}.
If the name \url{mail.example.com} %is part of the same zone, 
is managed by the same vulnerable to zone poisoning parent name server, the adversary may only update an \texttt{A/AAAA} record of an original \url{mail.example.com} with its own IP address, as explained earlier.
In both cases, the adversary can store the emails of the clients and forward them back %the emails back 
to the victim mail server.
%MK: we'll explain later: If the adversary wants to run a phishing website it can also update the \texttt{MX} record and bypass email challenge to validate the domain (see Section \ref{sec:domainvalidation} for more details). 

%and map its own IP address for the domain. In this case the traffic for his server will be redirected towards to the adversaries server. However, the victim or its customers would be able to detect the outage. If the adversary run a phishing website similar to the victim, he might be able to successfully steal customer information. Similarly, adversary will also be able to build a completely new sub-domain of our victim legitimate domain. He can then use the sub-domain to collect 




%\begin{itemize}
%\item redirect traffic for disruption
%\item add a subdomain and redirect info after stealing 
%\item add a subdomain to launch a phising attack 
%\item add a domain with domain verfication using certs to show as legitimate site 
%\end{itemize}
%\textbf{Add/Update MX record:} 
%\begin{itemize}
%\item redirect all mail to own servers to steal important information 
%\item add another mx record with subdomain to use in phishing scam 
%\item add an mx record with email server configured to by pass domain verification required for SSL certificates
%\end{itemize}


\subsubsection{Domain Shadowing Attack}

The adversary can spawn the subdomains for phishing and exploit kit attacks by adding new \texttt{A} and \texttt{AAAA} records to the vulnerable zone files or by adding a DNS delegation that provides a mapping between a new subdomain and a new child name server maintained by the adversary. % and its glue record.

\textbf{Addition of an \texttt{A/AAAA} record:} 
The adversary can add an \texttt{A/AAAA} record for a subdomain with an IP address pointing to, for example, the web server hosting a malware delivery or a phishing website (e.g.: \texttt{bankofamerica.example.com A 123.123.123.123} or \texttt{paypal.login.account.example.com A 123.123.123.123}). 
%under its control (\texttt{paypal.example.com A 123.123.123.123} or \texttt{bankofamerica.client.example.com A 123.123.123.123}) hosting a phishing website. %in the victims zone file. 
She can %spawn 
add unlimited number of subdomains and rotate between them to evade detection. 
%The adversary could use the business model of the parent domain and can set up subdomains for phishing attack.
%Furthermore, the attacker 
The adversary uses the reputation of the victim's domain to by-pass domain blacklisting services. %for the exploit kit attack. 
It can lead to adversely impact on the reputation of the victim's domain and business after the attack is discovered by %various
blacklisting services.
To increase the stealthiness of the attack and to prevent other classical detection methods like IP or websites blocking the adversary may use the fast-flux technique and quickly convert multiple subdomains to a wide range of IP addresses.
The victim would not be able to detect the attack unless it regularly checks the content of the zone file or is notified. 
However, the adversary can also update the \texttt{MX} record, as explained in the previous section, which can be useful in redirecting abuse emails to the attacker and therefore in further increasing the stealthiness of the~attack. 
The introduction of the General Data Protection Regulation (GDPR) makes the notification of the victim more challenging because the Registrant and Administrative Contact in the WHOIS, maintained independently from vulnerable name servers, is not publicly displayed anymore.

\textbf{Addition of an \texttt{NS} record:} %subdomain delegation:}
To spawn multiple subdomains and perform rotation between them the adversary needs to constantly update the zone file of the victim.
It might alert the victim because of multiple changes in the zone file of a vulnerable name server. 
To further improve the stealthiness of the domain shadowing  attack the adversary can add a delegation: the \url{example.com} zone to the \url{account.example.com} zone maintained by the child name server (\url{ns1.account.example.com}) and a corresponding glue record, providing a mapping between its name and the IP address of the malicious server.
Using this technique the adversary can spawn multiple subdomains on her server without performing any new updates to victim zone file.
For example, if the adversary adds a delegation for a domain
\url{account.example.com} % with the command %check this  \textit{nsupdate NS 10.10.10.10 accounts.example.com}. 
then she can %spawn 
add \url{paypal.account.example.com} or any other sibling domains in the zone file of %10.10.10.10 
the malicious server without any interaction with the victim parent name server. 

To provide an additional layer of redundancy and survivability of the phishing or malware infrastructure, the adversary may combine domain shadowing with double fast-flux.
The adversary may constantly change the name and IP addresses of the malicious child name server at the victim's parent zone file using zone poisoning as well as change the mapping between malware or phishing subdomains and their IP addresses in the malicious child name server. % controlled by the adversary.
The technique significantly increases the stealthiness of the attack without adding too much operational complexity to the adversary. 


\subsubsection{Compromising Domain Control Validation\label{sec:domainvalidation}}

% Why domain control validation is important .. motivation 
% WHich certs did we chose and why 
% Our methodology to bypass DCV 
% What does it add attack vectors 

In order to establish the authenticity of the domain and have an encrypted communication between the browser of the user and a website, a digital certificate is used. It offers confidentiality to the user as all the data is encrypted using the SSL/TLS certificate offered by the website owner. The user before sending the encrypted traffic to the website establishes the integrity of the certificate from a  Certificate Authority (CA). 

To acquire a digitally signed certificate from a reputed CA  the domain owner must establish its legitimacy. The process through which a  domain owner validates its ownership rights is known as Domain Control Validation (DCV). 

The three most common methods to establish the ownership are as follows. 
a) \textbf {Email validation} is one of the most well-known options to confirm domain ownership. During the certificate activation process, the domain owner selects one of the email addresses from either email address in whois records or domain-related emails. The domain related email addresses have generics in front of domains for instance for example.com  the valid addresses could be admin@example.com, administrator@example.com, postmaster@example.com, webmaster@example.com or hostmaster@example.com. Once the email address is selected the validation link is then sent to the requested email address for confirmation. b) \textbf{HTTP-based validation} requires to download a file from the CA and place it in the directory mentioned by the CA. The file must be accessible using HTTP request. For instance, in the case of example.com domain, the following URL should be accessible http://example.com/.well-known/pki-validation/filename.txt. This will enable CA to validate ownership of the domain owner. c) \textbf{DNS-based validation} requires you to create a certain CNAME record in the DNS settings of your domain. The CNAME contains a random string generated by the CA. If the URL is accessible by the CA validates the domain ownership. 


In our experimental setup we were able to by-pass domain control validation and were able to obtain certificates from two top Certificate Authorities (Let's encrypt and Comodo). %Majority of CA requires either email listed in SOA record or setup a challenge on the domain. Using our methodology the adversary can bypass both of the methods to obtain a legitimate certificate for the domain. 

\textbf{ Add MX record:} 
Adversary can also add mx record to victims zone file. It can be useful for particular places, a ) compromise email based challenge for domain control validation. %Adversary can also edit SOA records to get the abuse emails further avoid detection.


