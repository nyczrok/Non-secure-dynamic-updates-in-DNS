\section{Related work}


\subsection{DNSSEC}
\cite{goldberg2015nsec5} - The paper studes DNSSEC with NSEC and NSEC3 records, and shows that it inherently suffers from zone enumeration. They then propose a new construction that uses online public-key cryptography to solve the problem of DNSSEC zone enumeration - NSEC5
\cite{fiebig2017something} - We collect a new IPv6 addresses dataset spanning more than 5.8 million IPv6 addresses by exploiting DNS’ denial of existence semantics (NXDOMAIN).
\cite{kuhrer2015going} - research open DNS resolvers and found that many of them are malicious
\cite{moura2016anycast} - Evaluation of the DNS root event on Nov. 30 and Dec. 1, 2015.
\cite{schomp2014assessing} - Assessing DNS vulnerability to record injection with different attacks (but not dynamic updates)
\cite{pearce2017global} - Global measurement of dns manipulation
\cite{hao2018end} - DNS manipulation using dynaming mapping used by CDNs


\cite{van2015making} - studies DNSSEC reflection attack and blames RSA for that. They want to replace it by  elliptic curve cryptography (EC-DSA and EdDSA) are highly attractive for use in DNSSEC.
\cite{rasti2015temporal} - We introduce temporal lensing—a technique that concentrates a relatively low-bandwidth flood into a short, high-bandwidth pulse. Use this to perform DNS reflection attack. 
\cite{van2014dnssec} - Reflection attack using DNSSEC
\cite{macfarland2015characterizing} - Characterizing optimal DNS amplification attacks and effective mitigation
\cite{rossow2014amplification} - DNS reflections attack study using DNS DNSSEC and others. 




\cite{liang2013measuring} - measuring QoS of DNS root servers. 
\cite{osterweil2012behavior} - Behavior of DNS’Top Talkers, a .com/.net View
\cite{jonker2016measuring} - In this paper, we investigate the adoption of cloud-based DPSs worldwide. We focus on nine leading providers. Our outlook on adoption is made on the basis of active DNS measurements. We introduce a methodology that allows us, for a given domain name, to determine if traffic diversion to a DPS is in effect. It also allows us to distinguish various methods of traffic diversion and protection. 


Multiple studies focus on detection and clasification of  short-live domains automatically registrated by malwares \cite{vissers2017exploring,kountouras2016enabling,pereira2018dictionary,alrwais2014understanding,plohmann2016comprehensive,schuppen2018fanci}


\cite{zhang2014mismanagement} Utilizing Internet-scale measurements of DNS resolvers, BGP routers, and SMTP, HTTP, and DNS-name servers, we find there are thousands of networks where a large fraction of network services are misconfigured. 
\cite{dietrich2018investigating} - Measurements on network and DNS missconfiguration, why and how to protect.
\cite{borgolte2018cloud} - checking stale DNS records from old cloud services that were moved. Attackers can get those IP and get the traffic. 
\cite{lever2016domain} - analysing expired domains that are being registered by malicious attackers to highjack traffic. 
\cite{korczynski2016zone} - our previous paper
\cite{korczynski2017reputation} - Reputation metrics design to improve intermediary incentives for security of tlds



\cite{lever2013core} - try to find malware on mobile devices by analysing DNS. 
\cite{fukuda2015detecting} - Detecting Malicious Activity with DNS Backscatter (rDNS).





%%%%%%%%%%%%%%%%%% MEASUREMENTS %%%%%%%%%%%%%%%%%%%%%%%%%%%%%%%%%%%%%%%%
\subsection{Measurements}
\cite{schomp2013measuring} - measuring DNS client behaviour, caching etc. 
\cite{jones2016detecting} - We present techniques for detecting unauthorized DNS root servers in the Internet using primarily endpoint-based measurements from RIPE Atlas, supplemented with BGP routing announcements from RouteViews and RIPE RIS. 
\cite{fiebig2018rdns} - We observe that the share of non-authoritatively answerable IPv4 rDNS queries reduced since earlier studies and IPv6 rDNS has less non-authoritatively answerable queries than IPv4 rDNS. Furthermore, we compare passively collected datasets with actively collected ones, and we show that they enable observing the same effects in rDNS data. While highlighting oppor- tunities for future research, we find no immediate challenges to the use of rDNS as active and passive data-source for Internet measurement research.
%DNSSEC adoption/verification
\cite{lian2013measuring} - DNSSEC on client name resolution using an. A relatively small fraction of users are protected by DNSSEC-validating resolvers. and enabling DNSSEC measurably increases end-to-end resolution failures.
\cite{wander2013measuring} - checking whether a client is protected by DNSSEC validation. We applied our methodology over a period of 7 months collecting results from different data sources. 
\cite{chung2017understanding} - Looking at DNSSEC adoption
\cite{chung2017longitudinal} - A longitudinal, end-to-end view of the DNSSEC ecosystem
\cite{liu2018answering} - Understanding and characterizing interception of the DNS resolution path



\cite{zhu2015connection} - they propose to use TCP for DNS. That's a similar conclusion to ours. 






