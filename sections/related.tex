\section{Related work}

%%%%%%%%%%%%%%%%%% DNSSEC %%%%%%%%%%%%%%%%%%%%%%%%%%%%%%%%%%%%%%%%
\subsection{DNSSEC}
\cite{goldberg2015nsec5} - The paper studes DNSSEC with NSEC and NSEC3 records, and shows that it inherently suffers from zone enumeration. They then propose a new construction that uses online public-key cryptography to solve the problem of DNSSEC zone enumeration - NSEC5

\cite{van2015making} - DNSSEC adds digital signatures to the DNS, significantly increasing the size of DNS responses. This means DNSSEC is more susceptible to packet fragmentation and makes DNSSEC an attractive vector to abuse in amplification-based denial-of-service attacks. Additionally, key management policies are often complex. This makes DNSSEC fragile and leads to operational failures. In this paper, we argue that the choice for RSA as default cryptosystem in DNS- SEC is a major factor in these three problems. We show that  elliptic curve cryptography (EC-DSA and EdDSA) are highly attractive for use in DNSSEC, although they also have disadvantages. To address these, we have initiated research that aims to investigate the viability of deploying ECC at a large scale in DNSSEC.

\cite{lian2013measuring} - We have performed a large-scale measurement of the effects of DNSSEC on client name resolution us- ing an ad network to collect results from over 500,000 geographically-distributed clients. Our findings corroborate those of previous researchers in showing that a relatively small fraction of users are protected by DNSSEC-validating resolvers. And we show, for the first time, that enabling DNSSEC measurably increases end-to-end resolution failures. For every 10 clients that are protected from DNS tampering when a domain deploys DNSSEC, approximately one ordinary client (primarily in Asia) becomes unable to access the domain.

\cite{wander2013measuring} - In this paper we present a methodology to determine whether a client is protected by DNSSEC
validation. We applied our methodology over a period of 7 months collecting results from different data sources. After data cleaning, we gathered 131,320 results from 98,179 distinct IP addresses, out of which 4.8\% had validation enabled. The ratio varies significantly per country, with Sweden, the Czech Republic and the United States having the largest ratios of validating clients in the field.

\cite{rasti2015temporal} - We introduce temporal lensing—a technique that concentrates a relatively low-bandwidth flood into a short, high-bandwidth pulse. By leveraging existing DNS infrastructure, we experimentally explore lensing and the properties of the pulses it creates. We also show how attackers can use lensing to achieve peak bandwidths more than an order of magnitude greater than their up- load bandwidth. While formidable by itself in a pulsing DoS attack, we note how lensing can be compatibly combined with amplification attacks to potentially allow attackers to produce pulses with peak bandwidths orders of magnitude larger than their own.

\cite{vissers2017exploring} - This study extensively scrutinizes 14 months of registration data to identify large-scale malicious campaigns present in the .eu TLD. We explore the ecosystem and modus operandi of elaborate cybercriminal entities that recurrently register large amounts of domains for one-shot, malicious use. Although these malicious domains are short-lived, by in- corporating registrant information, we establish that at least 80.04\% of them can be framed in to 20 larger campaigns with varying duration and intensity. We further report on insights in the operational aspects of this business and observe, amongst other findings, that their processes are only partially automated. Finally, we apply a post-factum clustering process to validate the campaign identification process and to automate the ecosystem analysis of malicious registrations in a TLD zone.

\cite{chung2017understanding} - In this paper, we investigate the underlying reasons why DNSSEC adoption has been remarkably slow. We focus on registrars, as most TLD registries already support DNSSEC and registrars often serve as DNS operators for their customers. Our study uses large-scale, longitudinal DNS measurements to study DNSSEC adoption, coupled with experiences collected by trying to deploy DNSSEC on domains we purchased from leading domain name registrars and resellers. Overall, we find that a select few registrars are responsible for the (small) DNSSEC deployment today, and that many leading registrars do not support DNSSEC at all, or require customers to take cumbersome steps to deploy DNSSEC. Further frustrating deploy- ment, many of the mechanisms for conveying DNSSEC information to registrars are error-prone or present security vulnerabilities. Finally, we find that using DNSSEC with third-party DNS operators such as Cloudflare requires the domain owner to take a number of steps that 40\% of domain owners do not complete. Having identified several operational challenges for full DNSSEC deployment, we make recommendations to improve adoption.

\cite{van2014dnssec} - We perform a detailed measurement on a large dataset of DNSSEC-signed domains, covering 70\% (2.5 million) of all signed domains in operation today, and compare the potential for amplification attacks to a representative sample of domains without DNSSEC. At first glance, the outcome of these measurements confirms that DNSSEC indeed worsens the DDoS phenomenon. Closer examination, however, gives a more nuanced picture. DNSSEC really only makes the situation worse for one particular query type (ANY), for which responses may be over 50 times larger than the original query (and in rare cases up to 179x). We also discuss a number of mitigation strategies that can have immediate impact for operators and suggest future research directions with regards to these mitigation strategies.

%%%%%%%%%%%%%%%%%% MEASUREMENTS %%%%%%%%%%%%%%%%%%%%%%%%%%%%%%%%%%%%%%%%
\subsection{Measurements}
\cite{schomp2013measuring} - The paper presents methodologies for efficiently discovering the complex client-side DNS infrastructure. It further develops measurement techniques for isolating the behavior of the distinct actors in the infrastructure. Using these strategies, we study various aspects of the client-side DNS infrastructure and its behavior with respect to caching, both in aggregate and separately for different actors.


\cite{jones2016detecting} - We present techniques for detecting unauthorized DNS root servers in the Internet using primarily endpoint-based measurements from RIPE Atlas, supplemented with BGP routing announcements from RouteViews and RIPE RIS. The first approach analyzes the latency to the root server and the second approach looks for route hijacks.

\cite{fiebig2017something} - We collect a new IPv6 addresses dataset spanning more than 5.8 million IPv6 addresses by exploiting DNS’ denial of existence semantics (NXDOMAIN). This paper documents our efforts in obtaining new datasets of allocated IPv6 addresses, so others can avoid the obstacles we encountered.

\cite{fiebig2018rdns} - We observe that the share of non-authoritatively answerable IPv4 rDNS queries reduced since earlier studies and IPv6 rDNS has less non-authoritatively answerable queries than IPv4 rDNS. Furthermore, we compare passively collected datasets with actively collected ones, and we show that they enable observing the same effects in rDNS data. While highlighting oppor- tunities for future research, we find no immediate challenges to the use of rDNS as active and passive data-source for Internet measurement research.

\cite{liang2013measuring} - We surveyed the latency of upper DNS hierarchy from 19593 vantage points around the world to investigate the impact of uneven distribution of top level DNS servers on end-user latency. Our findings included: 1) generally top level DNS servers served Internet users efficiently, with median latency 20.26ms for root, 42.64ms for .com/.net, 39.07ms for .org; 2) quality of service was uneven, Europe and North America were the best while Africa and South America were 3 to 6 times worse; 3) most of the root servers performed well in Europe and North America, but only F, J, L roots showed low query latency in other continents; 4) query latency of F and L roots showed that only about 60\% resolvers were routed to the nearest anycast instances. We also revealed two problems that lead to constantly large query latency (6s-18s) for resolvers.

\cite{jonker2016measuring} - In this paper, we investigate the adoption of cloud-based DPSs worldwide. We focus on nine leading providers. Our outlook on adoption is made on the basis of active DNS measurements. We introduce a methodology that allows us, for a given domain name, to determine if traffic diversion to a DPS is in effect. It also allows us to distinguish various methods of traffic diversion and protection. For our analysis we use a long-term, large-scale data set that covers well over 50\% of all names in the global domain namespace, in daily snapshots, over a period of 1.5 years

%%%%%%%%%%%%%%%%%% SECURITY %%%%%%%%%%%%%%%%%%%%%%%%%%%%%%%%%%%%%%%%

\subsection{Security}
\cite{kountouras2016enabling} - Internet miscreants make extensive use of short-lived disposable domains to promote a large variety of threats and support their criminal network operations. we have created a system, Thales, that actively queries and collects records for massive amounts of domain names from various seeds. These seeds are collected from multiple public sources and, therefore, free of privacy concerns. The results of this effort will be opened and made freely available to the research community. With three case studies we demonstrate the detection merit that the collected active DNS datasets contain. We show that (i) more than 75\% of the domain names in public black lists (PBLs) appear in our datasets several weeks (and some cases months) in advance, (ii) existing DNS research can be implemented using only active DNS, and (iii) malicious campaigns can be identified with the signal provided by active DNS.


\cite{dietrich2018investigating} - We investigate the operators’ perspective on security misconfigurations to approach the human component of this class of security issues. We focus our analysis on system operators, as although they are the relevant actors managing the affected sys- tems, they have not yet received significant attention by prior research. We follow an inductive approach and apply a multi-step empirical methodology: (i) a qualitative study to understand how to approach the target group and measure the misconfiguration phenomenon, and (ii) a quantitative survey rooted in the qualitative data. We then provide the first analysis of system operators’ perspective on security misconfigurations, and we determine the factors that operators perceive as the root causes. Based on our findings, we provide practical recommendations on how to reduce security misconfigurations’ frequency and impact.


\cite{borgolte2018cloud} - we discover a substantial number of stale DNS records that point to available IP addresses in clouds, yet, are still actively attempted to be accessed. Often, these records belong to discontinued services that were previously hosted in the cloud. We demonstrate that it is practical, and time and cost efficient for attackers to allocate IP addresses to which stale DNS records point. An attacker can impersonate the service using a valid certificate trusted by all major operating systems and browsers. The attacker can then also exploit residual trust in the domain name for phishing, receiving and sending emails, or possibly distribute code to clients that load remote code from the domain (e.g., loading of native code by mobile apps, or JavaScript libraries by websites). An aggressive attacker could execute the attack in less than 70 seconds, well below common time-to-live (TTL) for DNS records. In turn, it means an attacker could exploit normal service migrations in the cloud to obtain a valid SSL certificate for domains owned and managed by others, and, worse, that she might not actually be bound by DNS records being (temporarily) stale, but that she can exploit caching instead.

\cite{lever2016domain} - Any individual that re-registers an expired domain implicitly inherits the residual trust associated with the domain’s prior use. We find that adversaries can, and do, use malicious re-registration to exploit domain ownership changes—undermining the security of both users and systems. In fact, we find that many seemingly disparate security problems share a root cause in residual domain trust abuse. With this study we shed light on the seemingly unnoticed problem of residual domain trust by measuring the scope and growth of this abuse over the past six years. To help address this problem, we propose a technical remedy and discuss several policy remedies. For the former, we develop Alembic, a lightweight algorithm that uses only passive observations from the Domain Name System (DNS) to flag potential domain ownership changes. We identify several instances of residual trust abuse using this algorithm, including an expired APT domain that could be used to revive existing infections.

\cite{kuhrer2015going} - Since several years, millions of recursive DNS resolvers are-deliberately or not—open to the public. This, however, is counter intuitive, since the operation of such openly accessible DNS resolvers is necessary in rare cases only. Furthermore, open resolvers enable both amplification DDoS and cache snooping attacks, and can be abused by attackers in multiple other ways. We thus find open recursive DNS resolvers to remain one critical phenomenon on the Internet. In this paper, we illuminate this phenomenon by analyzing it from two different angles. On the one hand, we study the landscape of DNS resolvers based on empirical data we collected for over a year. We analyze the changes over time and classify the resolvers according to device type and software version. On the other hand, we take the viewpoint of a client and measure the response authenticity of these resolvers. Besides legitimate redirections (e.g., to captive portals or router login pages), we find millions of resolvers to deliberately manipulate DNS resolutions (i.e., return bogus IP address information). To understand this threat in more detail, we systematically analyze non-legitimate DNS responses and reveal open DNS resolvers that manipulate DNS resolutions to censor communication channels, inject advertisements, serve malicious files, perform phishing, or redirect to other kinds of suspicious or malicious activities.

\cite{moura2016anycast} - This paper provides the first evaluation of several IP anycast services under stress with public data. Our subject is the Internet’s Root Domain Name Service, made up of 13 independently designed services (“letters”, 11 with IP anycast) run- ning at more than 500 sites. Many of these services were stressed by sustained traffic at 100x normal load on Nov. 30 and Dec. 1, 2015. We use public data for most of our analysis to examine how different services respond to stress, and identify two policies: sites may absorb attack traffic, containing the damage but reducing service to some users, or they may withdraw routes to shift both good and bad traffic to other sites. We study how these deployment policies resulted in different levels of service to different users during the events. We also show evidence of collateral damage on other services located near the attacks.

\cite{lever2013core} - Using DNS traffic collected over the course of three months from a major US cellular provider as well as a major US non- cellular Internet service provider, we identify the DNS domains looked up by mobile applications, and analyze information related to the Internet hosts pointed to by these domains. We make several important observations. The mobile malware found by the research community thus far appears in a minuscule number of devices in the network: 3,492 out of over 380 million (less than 0.0009\%) observed during the course of our analysis. This result lends credence to the argument that, while not perfect, mobile application markets are currently providing adequate security for the majority of mobile device users. Second, we find that users of iOS devices are virtually identically as likely to communicate with known low reputation domains as the owners of other mobile platforms, calling into question the conventional wisdom of one platform demonstrably providing greater security than another. Finally, we observe two malware campaigns from the upper levels of the DNS hierarchy and analyze the lifetimes and network properties of these threats. We also note that one of these campaigns ceases to operate long before the malware associated with it is discovered suggesting that network-based countermeasures may be useful in the identification and mitigation of future threats.

\cite{pereira2018dictionary} - In this paper, we design and implement a method called WordGraph for extracting dictionaries used by the Domain Generation Algorithms (DGAs) based solely on DNS traffic. Our result immediately gives us an efficient mechanism for detecting this elusive, new type of DGA, without any need for reverse engineering to extract dictionaries. Our experimental results on data from known Dictionary-AGDs show that our method can extract dictionary information that is embedded in the malware code even when the number of DGA domains is much smaller than that of legitimate domains, or when multiple dictionaries are present in the data. This allows our approach to detect Dictionary-AGDs in real traffic more accurately than state-of-the-art methods based on human defined features or featureless deep learning approaches.
