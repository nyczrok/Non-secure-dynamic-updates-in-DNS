\section{Descriptive statistics of vulnerable resources}
\subsection{Country overview}
\autoref{fig:map} presents geographic distribution of vulnerable resources. In highlighted countries, we detected at least one vulnerable server while the size of each circle is proportional to their number. We discovered vulnerable servers in 125 countries across the globe including all the countries in Europe, North America and a vast majority of countries in Asia, South America, Australia and Oceania. Relatively low number of vulnerable servers in the African region can be explained by low number of total DNS servers in the area \footnote{\url{https://labs.ripe.net/Members/emileaben/dns-root-server-transparency}}. \michal{We can probably get a better reference here}

The vast majority of vulnerable servers are located in Japan (716151), US (6442), South Korea (2922), Turkey (2701), Brazil (2080), Germany (1209), Taiwan (1135) and Canada (1047). 

\begin{figure}[!hbt]
\centering
\includegraphics[width=1\columnwidth]{map.png}
\caption{Countries}
\label{fig:map}
\end{figure}

\begin{figure}[!hbt]
\centering
\includegraphics[width=1\columnwidth]{heatmap.png}
\caption{Heatmap - Japan aligned with the US}
\label{fig:heatmap}
\end{figure}

\subsection{Per-AS statistics}
We continue by investigating distribution of vulnerable resources across different Autonomous Systems (AS). \autoref{fig:ip_pie} shows the number of vulnerable domains per authoritative server. 2 authoritative servers (119.82.8.252 and 119.82.8.251) are responsible for 80\% of vulnerable domains, while the 3rd server is responsible for 2.2\% of them. 

Those results are confirmed by per-AS analysis (\autoref{fig:ip_as_pie}). AS 33363 contains two main vulnerable authoritative servers and constitutes 83.9\% of vulnerable domains. 5 ASes containing the largest amount of vulnerable domains are responsible for 95.7\% of the total number of vulnerable resources. 

\begin{figure}[!hbt]
\centering
\includegraphics[width=.8\columnwidth]{ip_pie_anonymous.png}
\caption{IP addresses}
\label{fig:ip_pie}
\end{figure}

\begin{figure}[!hbt]
\centering
\includegraphics[width=.8\columnwidth]{ip_as_pie_anonymous.png}
\caption{AS}
\label{fig:ip_as_pie}
\end{figure}

\subsection{Per-CERT statistics}
\subsection{Per-TLD statistics}

\begin{figure}[!hbt]
\centering
\includegraphics[width=.8\columnwidth]{tld_pie.png}
\caption{TLD}
\label{fig:tld_pie}
\end{figure}

\subsection{Popularity of affected domains}

Among the vulnerable domains we find 5089 domains present on Alexa 1M list in a period between April 2015 and November 2018. The most popular affected domains reached 266th place on the list. Fig. \ref{fig:domains_cdf} presents cumulative distribution function of the positions reached  by the affected domains on the Alexa 1M list. 


\begin{figure}[!hbt]
\centering
\includegraphics[width=.8\columnwidth]{domains_cdf.png}
\caption{Popularity of affected domains - CDF}
\label{fig:domains_cdf}
\end{figure}
